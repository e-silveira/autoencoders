\documentclass{beamer}

\usefonttheme{serif}

\usepackage[portuguese]{babel}
\usepackage{graphicx}
\usepackage{caption}
\usepackage{ragged2e}

\usepackage{tikz}
\usetikzlibrary{positioning}

\title{Autoencoders}
\author{Eduardo de Medeiros da Silveira}
\institute{Universidade Federal de Santa Maria}
\date{}

\begin{document}

\frame{\titlepage}

\begin{frame}{Autoencoder}

\justifying

Um \emph{autoencoder} é uma rede neural que tenta copiar a sua entrada para a sua saída.

\begin{figure}

\centering

\begin{tikzpicture}[
roundnode/.style={circle, draw=black, minimum size=10mm}
]

\node[roundnode] (hidden) {$\boldsymbol{h}$};
\node[roundnode] (input) [below left=of hidden] {$\boldsymbol{x}$};
\node[roundnode] (output) [below right= of hidden] {$\boldsymbol{x'}$};

\draw[->] (input.north east) -- node[pos=0.35, above] {$f$} (hidden.south west);
\draw[->] (hidden.south east) -- node[pos=0.65, above] {$g$} (output.north west);

\end{tikzpicture}

\caption{
\justifying
Esquema geral de um \emph{autoencoder}, que mapeia uma entrada $\boldsymbol{x}$ para uma saída $\boldsymbol{x'}$, através de uma representação interna $\boldsymbol{h}$. 
O \emph{autoencoder} é composto por um codificador $f$ e por um decodificador $g$.
}

\end{figure}

\end{frame}

\begin{frame}{Código, ou representação latente}
    
\end{frame}

\end{document}
